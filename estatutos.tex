\documentclass[12pt]{article}
\usepackage[left=3.5cm, right=3.5cm, top=3cm, bottom=3cm]{geometry}
\usepackage[utf8]{inputenc}
\usepackage[section]{placeins}
\usepackage{enumitem}

\title{Estatutos de la asociación - Grupo de Usuarios de Linux de la Universidad Carlos III de Madrid}
%\author{}
\date{25 de enero de 2019}

\begin{document}
\maketitle
\thispagestyle{empty}
\newpage

\tableofcontents
\newpage


\section{CAPÍTULO I: DENOMINACIÓN, FINES, ACTIVIDADES, DOMICILIO Y ÁMBITO}


\subsection{Artículo 1.- Denominación}
Con la denominación de Grupo de Usuarios de Linux de la Universidad Carlos III de Madrid, se constituye una entidad sin ánimo de lucro, al amparo del artículo 22 CE, que se regirá por la Ley Orgánica 1/2002 de 22 de marzo, reguladora de derecho de asociación y normas concordantes y las que en cada momento le sean aplicables y por los Estatutos vigentes.

\subsection{Artículo 2.- Fines}
\begin{itemize}
    \item Divulgación y promoción de la informática y la tecnología haciendo énfasis en el uso del software libre.
    \item Divulgación y promoción de sistemas operativos libres basados en GNU/Linux.
    \item Divulgación y promoción de las cuatro libertades del software libre: derecho de publicación, uso, distribución y edición.
    \item Servir de medio de promoción e intercambio de experiencias del software libre entre alumnos y profesores de los diferentes niveles de enseñanza, especialmente universitaria.
    \item Realización de actividades, cursos, congresos, charlas e instalaciones, el acceso a los sitemas operativos y demás aplicaciones de software libre.
    \item Promoción del aprendizaje cooperativo y autodidacta sobre tecnología y software libre.
\end{itemize}

\subsection{Artículo 3.- Actividades}
Para el cumplimiento de estos fines, se realizarán las siguientes actividades:
\begin{itemize}
    \item Realización de charlas, talleres y competiciones.
    \item Creación y promoción de grupos de interés y secciones técnicas.
    \item Proyectos de desarrollo y documentación sobre software libre.
    \item Encuentros periódicos entre usuarios.
    \item Cualquier otra actividad lícita que sirva a los fines de esta asociación.
\end{itemize}

\subsection{Artículo 4.- Domicilio social}
La Asociación establece su domicilio social en (completo según se indica) Avenida de la Universidad, Nº 30, Campus de la Escuela Politécnica Superior de la Universidad Carlos III de Madrid, despacho 2.3.C05, localidad de Leganés, código postal 28911.

\subsection{Artículo 5.- Ámbito territorial}
El ámbito de actuación en el que se desarrollarán principalmente las actividades de la Asociación será el territorio de Leganés.


\section{CAPÍTULO II: ÓRGANOS DE LA ASOCIACIÓN}


\subsection{Artículo 6.- Órganos de gobierno y representación de la Asociación}
Los órganos de gobierno y representación de la Asociación son, respectivamente, la Asamblea General y la Junta Directiva.


\section{CAPÍTULO III: ASAMBLEA GENERAL}

\subsection{Artículo 7.- Naturaleza}
La Asamblea General es el órgano supremo de gobierno de la Asociación y estará integrada por todos los asociados.

\subsection{Artículo 8.- Reuniones}
Las reuniones de la Asamblea General se celebrarán, al menos, una vez al año, y de forma regular una vez al mes, teniendo el carácter de extraordinaria cuando lo solicite un número de asociados igual o superior al 10\%.

\subsection{Artículo 9.- Convocatorias}
Las convocatorias de las Asambleas Generales, tanto ordinarias como extraordinarias, se harán por escrito, expresando el lugar, día y hora de la reunión, así como el orden del día. Entre la convocatoria y el día señalado para la celebración de la Asamblea en primera convocatoria habrán de mediar al menos 3 días, pudiendo así mismo hacerse constar si procediera la fecha en que se reunirá la Asamblea en segunda convocatoria, sin que entre una convocatoria y otra pueda mediar un plazo inferior a 3 horas.

Por razones de urgencia, podrán reducirse los mencionados plazos.

\subsection{Artículo 10.- Quórum de validez de constitución y quórum de adopción de acuerdos}
Las Asambleas Generales, tanto ordinarias como extraordinarias, quedarán válida- mente constituidas cuando concurran a ellas, presentes o representados, al menos 1/3 de los asociados con derecho a voto. Los acuerdos se tomarán por mayoría simple de votos de las personas presentes o representadas, salvo en los supuestos de modificación de estatutos, disolución de la asociación, disposición o enajenación de bienes o remuneraciones de los miembros de la Junta Directiva, en los que será necesaria una mayoría cualificada del 60\% de votos de las personas presentes o representadas, decidiendo en caso de empate el voto de calidad del Presidente, o de quien haga las veces.

\subsection{Artículo 11.- Facultades de la Asamblea General}
Son facultades de la Asamblea General:
\begin{enumerate}[label=\alph*)]
    \item Nombramiento de la Junta Directiva y sus cargos, administradores y representantes, así como sus socios de honor.
    \item Examinar y aprobar los presupuestos anuales y las cuentas.
    \item Aprobar, en su caso, la gestión de la Junta Directiva.
    \item Fijar las cuotas ordinarias y extraordinarias.
    \item Acuerdo para constituir una federación de asociaciones o integrarse en alguna.
    \item Expulsión de socios a propuesta de la Junta Directiva.
    \item Solicitud de la declaracion de utilidad pública.
    \item Disposición y enajenación de bienes por mayoría cualificada.
    \item Aprobar el Reglamento de Régimen Interior.
    \item Remuneración, en su caso, de los miembros de la Junta Directiva.
    \item La modificación de los estatutos (convocada al efecto y aprobación de la mayoría cualificada).
    \item La disolución de la Asociación (convocada al efecto y aprovada por la mayoría cualificada).
\end{enumerate}


\section{CAPÍTULO IV: JUNTA DIRECTIVA}


\subsection{Artículo 12.- Naturaleza y composición}
La Junta Directiva es el órgano de representación que gestiona y representa los intereses de la asociación de acuerdo con las disposiciones y directivas de la Asamblea General. Estará formada por un Presidente, Secretario y en su caso un Vicepresidente, un Tesorero, un Coordinador de actividades y un Vocal, designados por la Asamblea General entre los asociados mayores de edad, en pleno uso de sus derechos civiles que no estén incursos en motivos de incompatibilidad legalmente establecidos.

Su mandato tendrá una duración de 1 año. El Presidente, Vicepresidente y Secretario de la Junta Directiva serán así mismo Presidente, Vicepresidente y Secretario de la Asociación y de la Asamblea General. Todos los cargos de la Junta Directiva se desempeñan de forma gratuita.

\subsection{Artículo 13.- Procedimiento para la elección y sustitución de miembros}
La elección de miembros de la Junta Directiva por la Asamblea General se realizará mediante una presentación de candidaturas a las que se les permitirá la adecuada difusión, con una antelación de 15 días a la celebración de la correspondiente reunión. En caso de ausencia o enfermedad de algún miembro de la Junta Directiva podrá ser suplido provisionalmente por otro de los componentes de ésta, previa predesignación por mayoría de sus miembros, salvo en el caso del Presidente que será sustituido por el Vicepresidente.

Los miembros de la Junta Directiva cesarán:
\begin{enumerate}[label=\alph*)]
    \item Por transcurso del período de sus mandatos.
    \item Por renuncia expresa.
    \item Por acuerdo de la Asemblea General.
\end{enumerate}

\subsection{Artículo 14.- Reuniones y quórum de constitución y adopción de acuerdos}
La Junta Directiva se reunirá previa convocatoria debiendo mediar al menos 3 días entre ésta y su celebración, cuantas veces lo determine su Presidente y a petición de 2/5 de sus miembros. Quedará constituida cuando asista la mitad más uno de sus miembros y para que sus acuerdos sean válidos deberán ser adoptados por mayoría de votos. En caso de empate, será de calidad del voto del Presidente o de quien haga sus veces.

\subsection{Artículo 15.- Facultades de la Junta Directiva}
Son facultades de la Junta Directiva:
\begin{enumerate}[label=\alph*)]
    \item Dirigir las actividades sociales y llevar la gestión económica y administrativa de la Asociación, acordando realizar los oportunos contratos y actos sin perjuicio de lo dispuesto en el artículo 11, apartado h).
    \item Ejecutar los acuerdos de la Asamblea General.
    \item Elaborar y someter la aprobación de la Asamblea General los presupuestos anuales y las cuentas.
    \item Elaborar en su caso el Reglamento de Régimen Interior.
    \item Resolver sobre la admisión de nuevos asociados.
    \item Nombrar delegados para alguna determinada actividad de la Asociación.
    \item Cualquiera otra facultad que no sea de la exclusiva competencia de la Asamblea General.
\end{enumerate}

\subsection{Artículo 16.- El Presidente}
El presidente tendrá las siguientes atribuciones:
\begin{enumerate}[label=\alph*)]
    \item Representar legalmente a la asociación ante toda clase de organismos públicos o privados.
    \item Convocar, presidir y levantar las sesiones que celebre la Asamblea General y la Junta Directiva.
    \item Dirigir las deliberaciones de una y otra.
    \item Ordenar pagos y autorizar con su firma los documentos, actas, y correspondencia.
    \item Adoptar cualquier medida urgente que la buena marcha de la Asociación aconseje, resulte necesaria o conveniente para el desarrollo de sus actividades, sin perjucio de dar cuenta posteriormente a la junta directiva.
\end{enumerate}

\subsection{Artículo 17.- El Vicepresidente}
El Vicepresidente sustituirá al Presidente en ausencia de éste, sea por enfermedad o cualquier otro motivo, y tendrá las mismas atribuciones que él.

Tendrá a su cargo la dirección de los trabajos puramente administrativos de la Asociación, expedirá certificados, llevará los ficheros y custodiará la documentación de la entidad, remitiendo en su caso, las comunicaciones a la Administración, con los requisitos pertinentes.

\subsection{Artículo 18.- El Secretario}
El Secretario tendrá a su cargo la dirección de los trabajos puramente administrativos de la Asociación, expedirá certificaciones, llevará los ficheros y custodiará la documentación de la entidad, remitiendo en su caso, las comunicaciones a la Administración, con los requisitos pertinentes.

\subsection{Artículo 19.- El Tesorero}
El Tesorero recaudará los fondos pertenecientes a la Asociación y dará cumplimiento a las órdenes de pago que expida el Presidente.

\subsection{Artículo 20.- El Vocal}
El Vocal tendrán las obligaciones propias de su cargo como miembro de la Junta Directiva y así como las que nazcan de las delegaciones o comisiones de trabajo que la propia junta le encomienda.


\section{CAPÍTULO V: LOS ASOCIADOS}


\subsection{Artículo 21.- Requisitos para asociarse}
Podrán pertenecer a la Asociación de forma libre y voluntaria, aquellas personas mayores de edad, con capacidad de obrar y no sujetas a condición que lo impida, que tengan interés en el desarrollo de los fines de la Asociación.

Así mismo podrán formar parte de la Asociación los menores no emancipados mayores de 14 años, con el consentimiento expreso de las personas que deban suplir su capacidad.

\subsection{Artículo 22.- Clases de Asociados}
Existirán las siguientes clases de asociados:
\begin{enumerate}[label=\alph*)]
    \item Fundadores, que serán aquellos que participen en el acto de constitución de la Asociación.
    \item De número, que serán los que ingresen después de la constitución de la Asociación.
    \item De honor, los que por su prestigio o por haber contribuido de modo relevante a los fines de la Asociación, se hagan acreedores de tal distinción.
    \item Asociados Juveniles: los mayores de 14 años y menores de 30, que formen o no parte de una Sección Juvenil.
\end{enumerate}

\subsection{Artículo 23.- Asociados activos e inactivos}
La condición de asociado activo permitirá el acceso a los derechos de asociado, en referencia a contabilidad de voto en la Asamblea General y demás reuniones invocadas al efecto de aprobar cualquier propuesta. La condición de asociado activo se adquirirá en el momento de ingreso como asociado en Libro de Socios de las Asociación y se perderá cuando:
\begin{enumerate}[label=\alph*)]
    \item Se produzca una ausencia de dos meses consecutivos a las reuniones mensuales o de grupos de trabajo interno, siempre que no se realice ninguna colaboración de forma telemática y el período de ausencia no se encuentre entre la finalización del ejercicio y el inicio del siguiente.
    \item Por petición propia del asociado en cuestión.
\end{enumerate}

La potestad de autorizar la inactividad de un asociado, por cualquiera de los motivos anteriores, recae en la Junta Directiva, la cual deberá constar de una mayoría del 60\% para efectuar dicha acción de inhabilitación de derechos del asociado.

La condición de activo se recuperará en el momento en el que el asociado inactivo en cuestión acuda a una reunión mensual o de grupo de trabajo.

La condición de inactividad no será reflejada en Libro de Socios, sino que será materia de RFI.

Las condiciones de actividad o inactividad son independientes del tipo de asociado.

\subsection{Artículo 24.- Causas de pérdida de la condición de asociado}
Se perderá la condición de asociado por alguna de las causas siguientes:
\begin{enumerate}[label=\alph*)]
    \item Por renuncia voluntaria, comunicada por escrito a la Junta Directiva.
    \item Por conducta incorrecta, o por desprestigiar la Asociación con hechos o palabras que perturben gravemente los actos organizados por la misma y la normal convivencia entre los asociados.
\end{enumerate}

\subsection{Artículo 25.- Derechos de los asociados}
Los asociados de número y los fundadores tendrán los siguientes derechos:
\begin{enumerate}[label=\alph*)]
    \item Participación en las actividades de la asociación y en los órganos de gobierno y representación.
    \item Ejercer el derecho de voto, así como asistir a la Asamblea General.
    \item Ser informado acerca de la composición de los órganos de gobierno y representación de la asociación, de su estado de cuentas y del desarrollo de su actividad.
    \item Ser escuchado con caŕacter previo a la adopción de medidas disciplinarias contra él.
    \item Impugnar los acuerdos de los órganos de la Asociación que estime contrarios a la ley o los estatutos.
    \item Hacer sugerencias a los miembros de la Junta Directiva en orden al mejor cumplimiento de los fines de la Asociación.
\end{enumerate}

Los asociados de honor y los menores de edad tendrán los mimsmos derechos salvo el de voto en la Asamblea General y el de participación en la Junta Directiva de la Asociación.

\subsection{Artículo 26.- Deberes de los asociados}
Los asociados tendrán las siguientes obligaciones:
\begin{enumerate}[label=\alph*)]
    \item Compartir las finalidades de la asociación y colaborar para la consecución de las mismas.
    \item Pagar las cuotas, derramas y otras aportaciones que, con arreglo a los estatutos, puedan corresponder a cada asociado, salvo los de honor.
    \item Cumplir el resto de obligaciones que resulten de las disposiciones estatuarias.
\end{enumerate}


\section{CAPÍTULO VI: RÉGIMEN DE FINANCIACIÓN, CONTABILIDAD Y DOCUMENTACIÓN}


\subsection{Artículo 27.- Obligaciones documentales y contables}
La asociación dispondrá de un relación actualizada de asociados. Asimismo, llevará una contabilidad donde quedará reflejada la imagen fiel del patrimonio, los resultados, la situación financiera de la entidad y las actividades realizadas. También dispondrá de un inventario actualizado de sus bienes.

En un Libro de Actas, figurarán las correspondientes a la reuniones que celebren sus órganos de gobierno y representación.

La Junta Directiva confeccionará todos los años un proyecto de presupuestos que presentará a la aprobación de la Asamblea General. Asimismo, presentará la liquidación de cuentas del año anterior. Todo lo relativo a la infraestructura de la organización y que no se considere en los presentes Estatutos, podrá ser regulado en el Reglamento de Funcionamiento Interno.

\subsection{Artículo 28.- Recursos económicos}
Los recursos económicos previstos para el desarrollo de los fines y actividades de la Asociación serán los siguientes:
\begin{enumerate}[label=\alph*)]
    \item Las cuotas de entrada, periódicas o extraordinarias, si las hubiese.
    \item Las subvenciones, donaciones, legados o herencias, que pudiera recibir por parte de los asociados o terceras personas.
    \item Cualquier otro recurso lícito.
\end{enumerate}

\subsection{Artículo 29.- Patrimonio Inicial y Cierre del Ejercicio}
La asociación carece de patrimonio inicial en el momento de fundación. El cierre del ejercicio asociativo será el día 15 de junio.


\section{CAPÍTULO VII: DISOLUCIÓN}


\subsection{Artículo 30.- Acuerdo de disolución}
La Asociación se disolverá:
\begin{enumerate}[label=\alph*)]
    \item Por voluntad de los asociados expresada mediante acuerdo de la Asamblea General, convocada al efecto por mayoría cualificada del 60\% de los asociados.
    \item Por imposibilidad de cumplir los fines previstos en los estatutos apreciados por acuerdo de la Asamblea General.
    \item Por sentencia judicial.
\end{enumerate}

\subsection{Artículo 31.- Comisión Liquidadora}
En caso de disolución, se nombrará una comisión liquidadora, la cual, una vez extinguidas las deudas y si existiese sobrante líquido, lo destinará para fines no lucrativos según lo acordado por la Asamblea General.

Los liquidadores tendrán las funciones que establecen los apartados 3 y 4 del artículo 18 de la Ley Orgánica 1/2002 de 22 de marzo.

En cualquier caso no está permitido el reparto del remanente entre los socios.

\subsection{Artículo 32.- Elaboración de RFI}
La elaboración de los RFI es competencia de la Junta Directiva en colaboración con la Sección Técnica o grupo correspondiente, en su totalidad o mediante un representante democráticamente elegido.
Para su aprobación se requiere mayoría absoluta favorable de la Junta Directiva.

No se podrán ampliar mediante RFI materias competencia de la Junta Directiva expuestas en estos Estatutos.

\end{document}